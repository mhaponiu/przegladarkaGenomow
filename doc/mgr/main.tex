\pdfoutput=1
\pdfcompresslevel=9
\pdfinfo
{
    /Author (Autor)
    /Title (Tytuł pracy magisterskiej)
    /Subject (Bioinformatyka)
    /Keywords (DNAASM)
}
%\documentclass[a4paper,polish,onecolumn,oneside,floatssmall,11pt,titleauthor,wide,openright]{mwrep}
%\usepackage[scale={0.7,0.8},paper=a4paper,twoside]{geometry}
\documentclass[a4paper,onecolumn,twoside,openright,11pt,wide,floatssmall]{mwrep}
% \usepackage{polish}
\usepackage{amsmath}
\usepackage{amsfonts}
\usepackage{amssymb}
\usepackage{amsthm}
\usepackage{bookman}
\usepackage{algorithm}
\usepackage{algpseudocode}
\usepackage{geometry}
\usepackage{placeins}
%\usepackage{slashbox}
\usepackage[utf8x]{inputenc}
\usepackage[T1]{fontenc}
\usepackage{footnote}
\usepackage{enumerate}
% \usepackage{fancyvrb}
% \usepackage{t1enc}
% \usepackage[pdftex, bookmarks]{hyperref}
\usepackage[pdftex, bookmarks=false]{hyperref}

\floatname{algorithm}{Algorytm}
\renewcommand{\algorithmicforall}{\textbf{for each}}

\usepackage{caption}
%\usepackage[vlined]{algorithm2e}

\def\url#1{{ \tt #1}}

\usepackage{listings}


\usepackage{fancyhdr}
\pagestyle{fancy}
\fancyhf{}
\fancyhead[CE,CO]{\leftmark}
\fancyfoot[LE,RO]{\thepage}
\usepackage[toc,page]{appendix}

\fancypagestyle{plain}{%
\fancyhf{}%
\fancyfoot[LE,RO]{\thepage}}

% marginesy
\textwidth\paperwidth
\advance\textwidth -55mm
\oddsidemargin-0.9in
\advance\oddsidemargin33mm
\evensidemargin0.9in
\advance\evensidemargin-33mm

\topmargin -1in
\advance\topmargin 25mm
\setlength\textheight{48\baselineskip}
\addtolength\textheight{\topskip}
\marginparwidth15mm

\clubpenalty=10000 % to kara za sierotki
\widowpenalty=10000 % nie pozostawia wdów
\brokenpenalty=10000 % nie dzieli wyrazów pomiędzy stronami
\sloppy

\tolerance4500
\pretolerance250
\hfuzz=1.5pt
\hbadness1450

\linespread{1.5} %interlinia
% ŻYWA PAGINA
\renewcommand{\chaptermark}[1]{\markboth{\scshape\small\bfseries \
#1}{\small\bfseries \ #1}}
\renewcommand{\sectionmark}[1]{\markboth{\scshape\small\bfseries\thesection.\
#1}{\small\bfseries\thesection.\ #1}}
\renewcommand{\headrulewidth}{0.5pt}
\renewcommand{\footrulewidth}{0.pt}
%%\pagestyle{uheadings}

\usepackage[pdftex]{color,graphicx}
\usepackage[polish]{babel}

% \textheight232mm
% \setlength{\textwidth}{\textwidth}
% \setlength{\oddsidemargin}{\evensidemargin}
% \setlength{\evensidemargin}{0.3cm}
\usepackage[sort, compress]{cite}

% definicje komand todo
\newcommand\todo[1]{\textcolor{red}{#1}}
\newcommand\mockup[1]{\textcolor{blue}{#1}}

%\usepackage{multibib}
%\newcites{bk,st,doc,web}{Książki i~artykuły,Standardy i~zalecenia,Dokumentacja produktów,Publikacje i~serwisy internetowe}

\theoremstyle{definition}
\newtheorem{defn}{Definicja}[chapter]
\newtheorem{conj}{Teza}[section]
\newtheorem{conjmain}{Teza}
\newtheorem{exmp}{Przykład}[chapter]

\theoremstyle{plain}% default
\newtheorem{thm}{Twierdzenie}[section]
\newtheorem{lem}[thm]{Lemat}
\newtheorem{prop}[thm]{Hipoteza}
\newtheorem*{cor}{Wniosek}

\theoremstyle{remark}
\newtheorem*{rem}{Uwaga}
\newtheorem*{note}{Uwaga}
\newtheorem{case}{Przypadek}[chapter]

\definecolor{ListingBackground}{rgb}{0.95,0.95,0.95}

\begin{document}

% kody źródłowe wplatane w tekst
\lstdefinestyle{incode}
{
basicstyle={\footnotesize},
keywordstyle={\bf\footnotesize\color{blue}},
commentstyle={\em\footnotesize\color{magenta}},
numbers=left,
stepnumber=5,
firstnumber=1,
numberfirstline=true,
numberblanklines=true,
numberstyle={\sf\tiny},
numbersep=10pt,
tabsize=2,
xleftmargin=17pt,
framexleftmargin=3pt,
framexbottommargin=2pt,
framextopmargin=2pt,
framexrightmargin=0pt,
showstringspaces=true,
backgroundcolor={\color{ListingBackground}},
extendedchars=true,
% title=\lstname,
captionpos=b,
% abovecaptionskip=1pt,
% belowcaptionskip=1pt,
frame=tb,
framerule=0pt,
}

% kody źródłowe z podpisem
\lstdefinestyle{outcode}
{
basicstyle={\footnotesize},
keywordstyle={\bf\footnotesize\color{blue}},
commentstyle={\em\footnotesize\color{magenta}},
numbers=left,
stepnumber=5,
firstnumber=1,
numberfirstline=true,
numberblanklines=true,
numberstyle={\sf\tiny},
numbersep=10pt,
tabsize=2,
xleftmargin=17pt,
framexleftmargin=3pt,
framexbottommargin=2pt,
framextopmargin=2pt,
framexrightmargin=0pt,
showstringspaces=true,
backgroundcolor={\color{ListingBackground}},
extendedchars=true,
% title=\lstname,
captionpos=b,
% abovecaptionskip=1pt,
% belowcaptionskip=1pt,
frame=tb,
framerule=0.1pt,
}

%\counterwithin{exmp}{chapter}
%\counterwithout{exmp}{section}
\renewcommand*\lstlistingname{Wydruk}
\renewcommand*\lstlistlistingname{Spis wydruków}

\pagenumbering{roman}
\renewcommand{\baselinestretch}{1.0}
\raggedbottom

% todo odkomentować strony tytulowe !
% 
\begin{titlepage}
    % Strona tytułowa


    % Życiorys
    \newpage\thispagestyle{empty}
    \begin{tabular}{p{5cm} p{12cm}}
    \begin{minipage}{5cm}
    \center
    %% \includegraphics[height=6.5cm,width=4.5cm]{img/ja.png}
    \end{minipage}
    &
    \begin{minipage}{12cm}
    \begin{flushleft}
    \par\noindent\vspace{0\baselineskip}
    \begin{tabular}[h]{l l}
    {\normalsize\it Specjalność:} & Informatyka -- \\
    & Systemy Informacyjno Decyzyjne \\
    \end{tabular}
    \par\noindent\vspace{1\baselineskip}
    \begin{tabular}[h]{l l}
    {\normalsize\it Data urodzenia:} & {\normalsize 03.04.1992}
    \end{tabular}
    \par\noindent\vspace{1\baselineskip}
    \begin{tabular}[h]{l l}
    {\normalsize\it Data rozpoczęcia studiów:} & {\normalsize 20.02.2016}%1 października 2012 r.}
    \end{tabular}
    \par\noindent\vspace{1\baselineskip}
    \end{flushleft}
    \end{minipage}
    \end{tabular}
    \vspace*{0\baselineskip}
    \begin{center}
	{\large\bfseries Życiorys}\par\bigskip
    \end{center}
	\mockup{
	Urodziłem się 3 kwietnia 1992 roku w~Kielcach. W~2008 roku rozpocząłem naukę w~II Liceum Ogólnokształcącym im. Jana Śniadeckiego w~Kielcach, w~klasie o~podstawie programowej w~zakresie rozszerzonym z~matematyki, informatyki, fizyki i~astronomii . W~2011 roku uzyskałem wykształcenie średnie oraz świadectwo dojrzałości... 
	}
    \indent

    \par
    \vspace{2\baselineskip}
    \hfill\parbox{15em}{{\small\dotfill}\\[-.3ex]
    \centerline{\footnotesize podpis studenta}}\par
    \vspace{1\baselineskip}
    \begin{center}
 	{\large\bfseries Egzamin dyplomowy} \par\bigskip\bigskip
    \end{center}
    \par\noindent\vspace{1.0\baselineskip}
    Złożył egzamin dyplomowy w dn. \dotfill
    \par\noindent\vspace{1.0\baselineskip}
    Z wynikiem \dotfill
    \par\noindent\vspace{1.0\baselineskip}
    Ogólny wynik studiów \dotfill
    \par\noindent\vspace{1.0\baselineskip}
    Dodatkowe wnioski i uwagi Komisji \dotfill
    \par\noindent\vspace{1.0\baselineskip}
    \dotfill

    % Streszczenie po polsku
    \newpage\thispagestyle{empty}
    \vspace*{2\baselineskip}
    \begin{center}
	{\large\bfseries Streszczenie}\par\bigskip
	\vspace*{2\baselineskip}
    \end{center}

    \itshape
    \todo{todo}
    \vspace*{3\baselineskip}

    \noindent{\bf Słowa kluczowe}: {\itshape \todo{todo.}}
    %\par
    %\vspace{4\baselineskip}
    
    % Streszczenie po angielsku
    \newpage\thispagestyle{empty}
    \vspace*{2\baselineskip}
    \begin{center}
	{\large\bfseries Abstract}\par\bigskip
	\vspace*{2\baselineskip}
    \end{center}
    \noindent{\bf Title}: {\itshape \todo{XXX}}\par
    \vspace*{1\baselineskip}
    \itshape
    \todo{XXX}
    \vspace*{3\baselineskip}

    \noindent{\bf Key words}: {\itshape \todo{XXX}.}

\end{titlepage}

% ex: set tabstop=4 shiftwidth=4 softtabstop=4 noexpandtab fileformat=unix filetype=tex spelllang=pl,en spell:


%\newpage\null\thispagestyle{empty}\newpage

% % Podziękowania
\newpage\thispagestyle{empty}



\tableofcontents
% \addcontentsline{toc}{chapter}{{Przedmowa1}{vii}}{vii}

% \chapter*{Spis tablic, rysunków i~wydruków}
% \listoftables
% \listoffigures
% \listofalgorithms
% \lstlistoflistings

%\setlength{\baselineskip}{7mm}
\newpage
\pagenumbering{arabic}
\raggedbottom{}
\setcounter{page}{1}



\chapter{Wstęp}
\label{section:wstep}

\section{Cel i zakres pracy}
\label{section:cel_i_zakres_pracy}

\subsection{Hipotezy}

\begin{enumerate}[I.]
	\item \textit{
		Można zaproponować strukturę bazy danych do przechowywania informacji opisujących genomy, taką że będzie ona elastyczna, łatwa w modyfikacji i~uniezależniona od semantyki przechowywanych struktur biologicznych.
		} \\
	\todo{krótki opis - duże dane, często nieznana struktura docelowa, wiele standardów, zanieczyszczone dane}
	
	\item \textit{
		Można dostarczyć abstrakcję widoków na dane genetyczne, umożliwiające analizę danych z różnych perspektyw w kontekście licznych zbiorów
		danych genetycznych.
		} \\
	\todo{różne podłoże semantyczne sekwencji, często sprzeczne niepełne informację, potrzeba różnych perspektyw na ten sam chromosom }
	
	\item \textit{
		Można dostarczyć aplikację do przechowywania i analizy genomów, która w~przystępny sposób umożliwi przeglądanie sekwencji genetyczych bez konieczności posiadania wydajnej maszyny klienckiej. 
		} \\
	\todo{duże dane, muszą być wydajne algorytmy, aplikacja lekka, intuicyjna, z dobrym api do dzielenia się z innymi zespołami}
	
\end{enumerate}

\subsection{Plan badań}
- generyczna baza danych \\
-genom ogorka, analiza danych sggw, produkcja chromosomów \\
-możliwość wdrożenia wydajnych implementacji algorytmów \\
-wygoda uzytkownika \\
-lekkie przeglądanie w skali makro i mikro \\

\section{Układ pracy}
\label{section:uklad_pracy}

\section{Przegląd literatury}
\label{section:przeglad_literatury}


\chapter{Biologiczna potrzeba}
\label{section:biologiny_wstep}

\section{Bioinformatyka w praktyce}
\subsection{Medycyna}
Posiadamy aktualnie techniczne możliwości aby stosować indywidualną wiedzę o indywidualnym genomie człowieka do projektowania indywidualnego leczenia. Dzięki medycynie spersonalizowanej jesteśmy w stanie w wielu przypadkach przewidywać ryzyko zachorowania na daną chorobę. Analizując genom, możemy zidentyfikować mutacje, które powodują, że niektóre białka działają w sposób inny niż powinny. Wykrycie takich przypadków nie oznacza, że człowiek od razu zachoruje, ponieważ istnieje bardzo rozbudowana sieć zależności i jeżeli jedno białko nie działa to być może w ten sam sposób funkcjonuje inne. Ma to natomiast istotne znaczenie przy projektowaniu leczenia. W sytuacji gdy dane białko nie działa, a standardowo leczymy chorobę modyfikując to białko, nabywamy cenną informacje, że leczenie należy przeprowadzić w inny sposób. Dążymy do tego aby minimalizować koszt i czas leczenia pacjentów, oraz zwiększyć wykrywalność chorób.

\subsection{Farmaceutyka}
\subsection{Kryminalistyka}
\subsection{Sądownictwo}
\subsection{Rolnictwo}
\subsection{Archeologia}

\section{Bioinformatyka w ujęciu algorytmicznym}
\subsection{Bazy danych}
\subsection{big data}
\subsection{uczenie maszynowe}
\subsection{metody optymalizacji}
\subsection{teoria grafów}

\section{Problemy bioinformatyki}
Bioinformatyka, jak każda dziedzina naukowa boryka się z pewnymi problemami. Problem dotyczący baz danych polega na tym, że wiele baz powstaje i często potem nic się z nimi nie dzieje. Z każdym dniem poznajemy tysiące nowych sekwencji, które są deponowane w cyfrowych przechowalniach. Te podstawowe bazy są utrzymywane i z powodzeniem wykorzystywane do naukowych doświadczeń. Problemem są bazy wtórne, które wykorzystują podstawowe eksperymenty. Gromadzą one zbiór informacji pochodzących z różnych miejsc.

-niekontynuowane projekty \\
-dane bazują na eksperymentach, błędy propagujące się \\
-wiele różnych standardów, niekompatybilne \\
-nieinformatyczni biolodzy \\
-sposoby finansowania nauki \\

\section{Centralny dogmat bioinformatyki}
-dna, rna, białko \\
-informacja genetyczna \\
-struktura molekularna \\
-funkcja biochemiczna \\
-fenotyp

\section{Rozwój bioinformatyki}
\subsection{wcześniej}
\subsection{obecnie}
\subsection{w przyszłości}

\section{Genom ogórka}
-sggw


\chapter{Projekt i implementacja}


\section{Architektura}

Przeglądarka została zaimplementowana w wielowarstwowej architekturze klient-serwer.
Kod został podzielony na moduły, aby jak najbardziej oddzielić od siebie niezależne fragmenty aplikacji zachowując przy tym dobre praktyki programistyczne.
Rozróżniamy w pracy 3 główne warstwy odpowiadające za:
\begin{itemize}
	\item prezentację
	\item obsługę danych
	\item logikę biznesową
\end{itemize}

\subsection*{BioWeb}
\todo{todo - bioweb, rysunek - model aplikacji trójwarstwowej}

\subsection*{Klient, warstwa prezentacji}
Moduł klienta znajdujący się w warstwie prezentacji możemy zaliczyć do grupy klientów cienkich.
Odpowiada on za ilustrowanie danych pobranych z serwera wykonując operacje renderowania elementów graficznych oraz zapewnia interaktywną komunikację użytkownika z systemem.
W obecnych czasach, praktycznie każdy komputer stacjonarny a nawet urządzenia mobilne posiadają stosunkowo dużą moc obliczeniową pozwalającą na proste przetwarzanie.
Realizując projekt, zastosowano podejście, w którym odpowiedzialność za operacje takie jak walidacja danych, generowanie rysunków czy animacje pozostają na barkach modułu klienckiego. 
Tym samym, odciążony zostaje serwer, przez co podwyższamy jego możliwości wydajnościowe co sprowadza się do zwiększenia potencjalnej liczby jednocześnie obsługiwanych użytkowników.

Do implementacji warstwy prezentacji zostały użyte technologie webowe, wykorzystujące jako środowisko uruchomieniowe przeglądarkę internetową. Użytkownikowi wystarczy jedynie maszyna z dostępem do internetu i nowoczesną przeglądarką internetową wpierająca HTML5. 
Klient może być uruchomiony na dowolnym popularnym systemie operacyjnym \mbox{(Linux/Mac/Windows)} i zwolniony jest potrzeby aktualizowania aplikacji - wszystkie moduły ładowane są podczas inicjacji przeglądarki genomu.

\section{Wykorzystane technologie}

\section{Struktura projektu}

\section{Baza danych}

\section{Algorytmy}

\section{Sekwencje ogórka}

\section{Interfejs użytkownika}

\section{Instrukcja uruchomienia aplikacji}



\chapter{Wyniki}

\section{Baza danych}
-znaczenie danych

\section{Sekwencje ogórka}
\todo{tabela z wynikami \ref{tab:genome_stat}}

\begin{table}
	\centering
	\begin{tabular}{|c||r|r|r|c|} \hline
		\textbf{Chromosom}    & Długość [Mbp] & Liczba & Zmapowana & Zmapowana \\
		& & zmapowanych contigów & długość [bp] & długość [\%] \\
		\hline
		\textbf{1}             & 55                 &  22                 &  32970425          & 59.9\%               \\
		\textbf{2}             & 44                 &  13                 &  23992470          & 52.2\%               \\
		\textbf{3}             & 65                 &  14                 &  39532546          & 60.8\%               \\
		\textbf{4}             & 61                 &  15                 &  24781874          & 40.6\%               \\
		\textbf{5}             & 49                 &  22                 &  26573846          & 54.2\%               \\
		\textbf{6}             & 42                 &  19                 &  29507078          & 70.2\%               \\
		\textbf{7}             & 52                 &  11                 &  18951584          & 36.4\%               \\
		\hline
		$\mathbf{\Sigma}$      &368                 & 116                 & 196309823          &  53.3\%              \\
		\hline
	\end{tabular}
	\caption{Chromosome statistic for mapped contigs of Nothern European cucumber genome.
		The length of the chromosomes and length of the genome described in~Chen's studies\cite{article:reevaluation_in_cucumber} was used.}
	\label{tab:genome_stat}
\end{table}

\section{API}

\section{Algorytmy}
Porównanie ze sobą dwóch sekwencji nie może polegać na zwykłej analizie ciągów tekstowych, ponieważ porównując ciągi musimy brać pod uwagę ich podłoże ewolucyjne. Chcemy sprawdzić na ile podobna jest jedna sekwencja do drugiej, analizując możliwość ewolucyjnego przekształcenia pierwszej w drugą. 
\todo{porównanie, komentarz}

\begin{figure}[h]
	\centering
	\includegraphics[width=1\textwidth]{img/sm-sekwencja-zmienna.png}
	\caption{\todo{opis - zmienna sekwencja}}
	\label{img:sm-sekwencja-zmienna}
\end{figure}

\begin{figure}[h]
	\centering
	\includegraphics[width=1\textwidth]{img/sm-wzorzec-zmienny.png}
	\caption{\todo{opis - zmienny wzorzec}}
	\label{img:sm-wzorzec-zmienny}
\end{figure}

\begin{figure}[h]
	\centering
	\includegraphics[width=1\textwidth]{img/kmp-vs-bm.png}
	\caption{\todo{opis - BMP vs KMP}}
	\label{img:kmp-vs-bm}
\end{figure}

\chapter{Podsumowanie}
\label{section:podsumowanie}

% \chapter{Opis algorytmów}
\label{ch:algorithms}



% \input{design}

% \chapter{Badania}
\label{section:badania}



% \input{summary}

%\chapter*{Bibliografia}
\nocite{*}
\bibliographystyle{plplain}
%\bibliographystylebk{plplain}
%\bibliographystylest{plplain}
%\bibliographystyledoc{plplain}
% \bibliographystyleweb{plplain}
%\bibliographybk{BIB/books}
%\bibliographyst{BIB/books}
%\bibliographydoc{BIB/books}
% \bibliographyweb{BIB/books}

\bibliography{bib/mgr}

\appendix

%% \input{spis_rysunkow}

%% \input{spis_tabel}

%\input {spis_zalacznikow}

% \chapter{Instrukcje użytkownika}
\label{instructions}



\end{document}

% ex: set tabstop=4 shiftwidth=4 softtabstop=4 noexpandtab fileformat=unix filetype=tex spelllang=pl,en spell:

