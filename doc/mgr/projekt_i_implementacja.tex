\chapter{Projekt i implementacja}


\section{Architektura}

Przeglądarka została zaimplementowana w wielowarstwowej architekturze klient-serwer.
Kod został podzielony na moduły, aby jak najbardziej oddzielić od siebie niezależne fragmenty aplikacji zachowując przy tym dobre praktyki programistyczne.
Rozróżniamy w pracy 3 główne warstwy odpowiadające za:
\begin{itemize}
	\item prezentację
	\item obsługę danych
	\item logikę biznesową
\end{itemize}

\subsection*{BioWeb}
\todo{todo - bioweb, rysunek - model aplikacji trójwarstwowej}

\subsection*{Klient, warstwa prezentacji}
Moduł klienta znajdujący się w warstwie prezentacji możemy zaliczyć do grupy klientów cienkich.
Odpowiada on za ilustrowanie danych pobranych z serwera wykonując operacje renderowania elementów graficznych oraz zapewnia interaktywną komunikację użytkownika z systemem.
W obecnych czasach, praktycznie każdy komputer stacjonarny a nawet urządzenia mobilne posiadają stosunkowo dużą moc obliczeniową pozwalającą na proste przetwarzanie.
Realizując projekt, zastosowano podejście, w którym odpowiedzialność za operacje takie jak walidacja danych, generowanie rysunków czy animacje pozostają na barkach modułu klienckiego. 
Tym samym, odciążony zostaje serwer, przez co podwyższamy jego możliwości wydajnościowe co sprowadza się do zwiększenia potencjalnej liczby jednocześnie obsługiwanych użytkowników.

Do implementacji warstwy prezentacji zostały użyte technologie webowe, wykorzystujące jako środowisko uruchomieniowe przeglądarkę internetową. Użytkownikowi wystarczy jedynie maszyna z dostępem do internetu i nowoczesną przeglądarką internetową wpierająca HTML5. 
Klient może być uruchomiony na dowolnym popularnym systemie operacyjnym \mbox{(Linux/Mac/Windows)} i zwolniony jest potrzeby aktualizowania aplikacji - wszystkie moduły ładowane są podczas inicjacji przeglądarki genomu.

\section{Wykorzystane technologie}

\section{Struktura projektu}

\section{Baza danych}

\section{Algorytmy}

\section{Sekwencje ogórka}

\section{Interfejs użytkownika}

\section{Instrukcja uruchomienia aplikacji}

