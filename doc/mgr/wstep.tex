
\chapter{Wstęp}
\label{section:wstep}

\section{Cel i zakres pracy}
\label{section:cel_i_zakres_pracy}

\subsection{Hipotezy}

\begin{enumerate}[I.]
	\item \textit{
		Można zaproponować strukturę bazy danych do przechowywania danych opisujących genomy, taką że będzie ona elastyczna, łatwa w modyfikacji, uniezależniona od semantyki przechowywanych struktur biologicznych.
		} \\
	\todo{krótki opis}
	
	\item \textit{
		Można dostarczyć abstrakcję widoków na dane genetyczne, umożliwiające analizę danych z różnych perspektyw w kontekście licznych zbiorów
		danych genetycznych.
		} \\
	\todo{krótki opis}
	
	\item \textit{
		Można dostarczyć aplikację do przechowywania i analizy genomów, która w~przystępny sposób umożliwi przeglądanie sekwencji genetyczych bez konieczności posiadania wydajnej maszyny klienckiej. 
		} \\
	\todo{krótki opis}
	
\end{enumerate}

\subsection{Plan badań}
-genom ogorka \\
-baza danych generycza \\
-algorytmy \\
-wygoda uzytkownika \\
-wydajne przegladanie \\

\section{Układ pracy}
\label{section:uklad_pracy}