
\chapter{Wstęp}
\label{section:wstep}

\section{Cel i zakres pracy}
\label{section:cel_i_zakres_pracy}

\subsection{Hipotezy}

\begin{enumerate}[I.]
	\item \textit{
		Można zaproponować strukturę bazy danych do przechowywania informacji opisujących genomy, taką że będzie ona elastyczna, łatwa w modyfikacji i~uniezależniona od semantyki przechowywanych struktur biologicznych.
		} \\
	\todo{krótki opis - duże dane, często nieznana struktura docelowa, wiele standardów, zanieczyszczone dane}
	
	\item \textit{
		Można dostarczyć abstrakcję widoków na dane genetyczne, umożliwiające analizę danych z różnych perspektyw w kontekście licznych zbiorów
		danych genetycznych.
		} \\
	\todo{różne podłoże semantyczne sekwencji, często sprzeczne niepełne informację, potrzeba różnych perspektyw na ten sam chromosom }
	
	\item \textit{
		Można dostarczyć aplikację do przechowywania i analizy genomów, która w~przystępny sposób umożliwi przeglądanie sekwencji genetyczych bez konieczności posiadania wydajnej maszyny klienckiej. 
		} \\
	\todo{duże dane, muszą być wydajne algorytmy, aplikacja lekka, intuicyjna, z dobrym api do dzielenia się z innymi zespołami}
	
\end{enumerate}

\subsection{Plan badań}
- generyczna baza danych \\
-genom ogorka, analiza danych sggw, produkcja chromosomów \\
-możliwość wdrożenia wydajnych implementacji algorytmów \\
-wygoda uzytkownika \\
-lekkie przeglądanie w skali makro i mikro \\

\section{Układ pracy}
\label{section:uklad_pracy}

\section{Przegląd literatury}
\label{section:przeglad_literatury}
